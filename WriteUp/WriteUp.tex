\documentclass[12pt,preprint, authoryear]{elsarticle}

\usepackage{lmodern}
%%%% My spacing
\usepackage{setspace}
\setstretch{1.2}
\DeclareMathSizes{12}{14}{10}{10}

% Wrap around which gives all figures included the [H] command, or places it "here". This can be tedious to code in Rmarkdown.
\usepackage{float}
\let\origfigure\figure
\let\endorigfigure\endfigure
\renewenvironment{figure}[1][2] {
    \expandafter\origfigure\expandafter[H]
} {
    \endorigfigure
}

\let\origtable\table
\let\endorigtable\endtable
\renewenvironment{table}[1][2] {
    \expandafter\origtable\expandafter[H]
} {
    \endorigtable
}


\usepackage{ifxetex,ifluatex}
\usepackage{fixltx2e} % provides \textsubscript
\ifnum 0\ifxetex 1\fi\ifluatex 1\fi=0 % if pdftex
  \usepackage[T1]{fontenc}
  \usepackage[utf8]{inputenc}
\else % if luatex or xelatex
  \ifxetex
    \usepackage{mathspec}
    \usepackage{xltxtra,xunicode}
  \else
    \usepackage{fontspec}
  \fi
  \defaultfontfeatures{Mapping=tex-text,Scale=MatchLowercase}
  \newcommand{\euro}{€}
\fi

\usepackage{amssymb, amsmath, amsthm, amsfonts}

\def\bibsection{\section*{References}} %%% Make "References" appear before bibliography


\usepackage[round]{natbib}

\usepackage{longtable}
\usepackage[margin=2.3cm,bottom=2cm,top=2.5cm, includefoot]{geometry}
\usepackage{fancyhdr}
\usepackage[bottom, hang, flushmargin]{footmisc}
\usepackage{graphicx}
\numberwithin{equation}{section}
\numberwithin{figure}{section}
\numberwithin{table}{section}
\setlength{\parindent}{0cm}
\setlength{\parskip}{1.3ex plus 0.5ex minus 0.3ex}
\usepackage{textcomp}
\renewcommand{\headrulewidth}{0.2pt}
\renewcommand{\footrulewidth}{0.3pt}

\usepackage{array}
\newcolumntype{x}[1]{>{\centering\arraybackslash\hspace{0pt}}p{#1}}

%%%%  Remove the "preprint submitted to" part. Don't worry about this either, it just looks better without it:
\makeatletter
\def\ps@pprintTitle{%
  \let\@oddhead\@empty
  \let\@evenhead\@empty
  \let\@oddfoot\@empty
  \let\@evenfoot\@oddfoot
}
\makeatother

 \def\tightlist{} % This allows for subbullets!

\usepackage{hyperref}
\hypersetup{breaklinks=true,
            bookmarks=true,
            colorlinks=true,
            citecolor=blue,
            urlcolor=blue,
            linkcolor=blue,
            pdfborder={0 0 0}}


% The following packages allow huxtable to work:
\usepackage{siunitx}
\usepackage{multirow}
\usepackage{hhline}
\usepackage{calc}
\usepackage{tabularx}
\usepackage{booktabs}
\usepackage{caption}


\newenvironment{columns}[1][]{}{}

\newenvironment{column}[1]{\begin{minipage}{#1}\ignorespaces}{%
\end{minipage}
\ifhmode\unskip\fi
\aftergroup\useignorespacesandallpars}

\def\useignorespacesandallpars#1\ignorespaces\fi{%
#1\fi\ignorespacesandallpars}

\makeatletter
\def\ignorespacesandallpars{%
  \@ifnextchar\par
    {\expandafter\ignorespacesandallpars\@gobble}%
    {}%
}
\makeatother

\newenvironment{CSLReferences}[2]{%
}

\urlstyle{same}  % don't use monospace font for urls
\setlength{\parindent}{0pt}
\setlength{\parskip}{6pt plus 2pt minus 1pt}
\setlength{\emergencystretch}{3em}  % prevent overfull lines
\setcounter{secnumdepth}{5}

%%% Use protect on footnotes to avoid problems with footnotes in titles
\let\rmarkdownfootnote\footnote%
\def\footnote{\protect\rmarkdownfootnote}
\IfFileExists{upquote.sty}{\usepackage{upquote}}{}

%%% Include extra packages specified by user

%%% Hard setting column skips for reports - this ensures greater consistency and control over the length settings in the document.
%% page layout
%% paragraphs
\setlength{\baselineskip}{12pt plus 0pt minus 0pt}
\setlength{\parskip}{12pt plus 0pt minus 0pt}
\setlength{\parindent}{0pt plus 0pt minus 0pt}
%% floats
\setlength{\floatsep}{12pt plus 0 pt minus 0pt}
\setlength{\textfloatsep}{20pt plus 0pt minus 0pt}
\setlength{\intextsep}{14pt plus 0pt minus 0pt}
\setlength{\dbltextfloatsep}{20pt plus 0pt minus 0pt}
\setlength{\dblfloatsep}{14pt plus 0pt minus 0pt}
%% maths
\setlength{\abovedisplayskip}{12pt plus 0pt minus 0pt}
\setlength{\belowdisplayskip}{12pt plus 0pt minus 0pt}
%% lists
\setlength{\topsep}{10pt plus 0pt minus 0pt}
\setlength{\partopsep}{3pt plus 0pt minus 0pt}
\setlength{\itemsep}{5pt plus 0pt minus 0pt}
\setlength{\labelsep}{8mm plus 0mm minus 0mm}
\setlength{\parsep}{\the\parskip}
\setlength{\listparindent}{\the\parindent}
%% verbatim
\setlength{\fboxsep}{5pt plus 0pt minus 0pt}



\begin{document}



\begin{frontmatter}  %

\title{Data Science Methods for Economics and Finance 871 Final Project:
Predicting Chess Outcomes}

% Set to FALSE if wanting to remove title (for submission)




\author[Add1]{Wesley Williams\footnote{You can find my code on my Github
  page: \url{https://github.com/wjwilliams/MLPROJ/tree/main/WriteUp}}}
\ead{21691126@sun.ac.za}





\address[Add1]{Stellenbosch University, South Africa}


\begin{abstract}
\small{
Chess is one of the most complicated games in the world. Humans on the
other hand are not so complicated and we are the ones playing the game.
This made me ask the question: ``Is it possible to predict a game of
chess based on just on data on the players before the game has started
and just the openings used?'' I employ an extreme gradient boosting
model to answer this question and find that I can only predict the
outcome with just over 60\% accuracy, which is double the chances of
just guessing.
}
\end{abstract}

\vspace{1cm}





\vspace{0.5cm}

\end{frontmatter}

\setcounter{footnote}{0}


\renewcommand{\contentsname}{Table of Contents}
{\tableofcontents}

%________________________
% Header and Footers
%%%%%%%%%%%%%%%%%%%%%%%%%%%%%%%%%
\pagestyle{fancy}
\chead{}
\rhead{}
\lfoot{}
\rfoot{\footnotesize Page \thepage}
\lhead{}
%\rfoot{\footnotesize Page \thepage } % "e.g. Page 2"
\cfoot{}

%\setlength\headheight{30pt}
%%%%%%%%%%%%%%%%%%%%%%%%%%%%%%%%%
%________________________

\headsep 35pt % So that header does not go over title




\newpage

\hypertarget{introduction}{%
\section{\texorpdfstring{Introduction
\label{Introduction}}{Introduction }}\label{introduction}}

Chess, often considered one of the most popular games globally, has
experienced a surge in popularity in recent years, partly fueled by its
portrayal in popular media such as the acclaimed TV show ``Queen's
Gambit'' and the movie ``Pawn.'' These portrayals, although
fictionalized to varying degrees, shed light on the remarkable life and
achievements of renowned chess player Bobby Fischer. The story of Bobby
Fischer is both inspirational and incredibly sad. During the Cold War,
the United States (US) and the Soviet Union (USSR) engaged in a fierce
competition to assert their global dominance. This rivalry extended
beyond conventional arenas like the space race and nuclear arms race,
encompassing intellectual pursuits such as chess. Chess was revered as
the game of the intellectual elite, and both nations sought to establish
themselves as the preeminent force in this domain, symbolizing their
intellectual superiority. The USSR dominated until Fischer's victory
against world champion Boris Spassky, symbolizing a small triumph for
the US in the Cold War context. This illustrates that chess extends
beyond a mere board game, carrying significant cultural and geopolitical
implications. While the stakes are not as high in the modern era, chess
is still seen as the epitome of intellect.

Chess can be broken down into three stages of the game: the opening, the
middlegame and the endgame. The opening represents the different
strategies of getting all of your pieces into the most optimal positions
on the board to both attack your opponents pieces and defend your own.
This is the key part of a chess game that will be investigated in the
paper. I want to determine if the outcome of a chess game can be
predicted using only the information available about the game and
players before the game begins and the openings employed by the players.
The use of machine learning is crucial in this analysis due to the
complexity of the game. I use an extreme gradient boosting model
(XGBOOST) with three potential outcomes (white winning, black winning or
a draw). The outcome is not binary therefore I cannot use a normal
ordinary least squares as comparison and so I use the probability of
guessing as the baseline. I also subset the data by ELO rating to assess
whether it is easier to predict weaker or stronger players. I find that
the opening is slightly more important for weaker players but the model
is not as accurate as when using the entire sample. The rest of the
paper is structured as follows, in section 2 I describe the data section
3 explores and visualizes the data, section 4 explains the methodology,
section 5 presents the results and section 6 concludes.

\hypertarget{data}{%
\section{Data}\label{data}}

The data was obtained from Kaggle and it includes over 20,000 games
played on the online chess platform ``lichess''. The dataset includes 16
variables but there are only seven variables of interest: 1) the winner
given as white, black or draw; 2) time and increment code which details
the time control as the base time and the additional time per move; 3)
white's ELO rating; 4) black's ELO rating; 5) moves which are all the
moves in the game given in chess notation; 6) opening eco which is a
code that represents the opening that was played and 7) opening ply
which represents the number of moves played in the opening that
corresponds to chess theory. An additional variable of rating difference
was engineered from the perspective of white, which is just white's
rating minus black's rating.

\hypertarget{factor-engineering}{%
\subsection{Factor engineering}\label{factor-engineering}}

Some of the variables need to be wrangled to be used in the model. All
of the numerical variables are sufficient for the model but the
character variables need to be engineered to be included. Firstly, I am
only interested in the first five moves of the game, I therefore expand
the moves variable so that I have a variable for each of the first five
moves played by each player and disregard the rest. The moves are then
converted from chess notation to a unique numerical factor for each
piece moving to each co-ordinate on the board. Secondly, I separate the
increment variable into the base time and the increment for each move
and ensure both are numerical factors. Lastly, I assign a unique
numerical value to each of the unique openeing codes to ensure
comparability between the training as testings
samples\footnote{I did attempt one-hot encoding but I had issues with the differences in lengths between the training and testing samples}.
All the variables are therefore sufficiently engineered to be used in
the model.

\hypertarget{exploratory-data-analysis}{%
\section{Exploratory Data Analysis}\label{exploratory-data-analysis}}

In this section I attempt to explore and visualize the data to gain
insights into the patterns that emerge with respect all the features and
the the target.

\begin{figure}[H]

{\centering \includegraphics{WriteUp_files/figure-latex/commove1-1} 

}

\caption{Most Common Opening Moves by Colour\label{Figure1}}\label{fig:commove1}
\end{figure}

\begin{figure}[H]

{\centering \includegraphics{WriteUp_files/figure-latex/firstmovepropwins-1} 

}

\caption{Proportions of Outcomes by White First Move\label{Figure2}}\label{fig:firstmovepropwins}
\end{figure}

Figure 3.1 shows the most common First move for each colour. It is no
surprise that central pawn moves are the most common as controlling the
center of the board is instrumental in the opening phase of a chess
game. Figure 3.2 shows the outcome proportions for white's first move
and it shows that if white claims the center they gain an advantage and
black needs to respond. The move e3 instead allows black to claim the
center and white loses its first mover advantage. This highlights that
mistakes early on in the openings have consequences that last throughout
the entire game.

\begin{figure}[H]

{\centering \includegraphics{WriteUp_files/figure-latex/Descript3wins-1} 

}

\caption{Outcomes by Time Controls\label{Figure3}}\label{fig:Descript3wins}
\end{figure}

Figure 3.3 shows the differences in outcome according to different time
controls. It presents the top 5 time controls that are played with and
without time increments. There are no significant differences in wins
between whether a game has time increments or not. The only real
difference is that a draw is more likely with larger time controls and
no time increments.

\begin{figure}[H]

{\centering \includegraphics{WriteUp_files/figure-latex/Desc4commonopen-1} 

}

\caption{Most Popular Openings\label{Figure4}}\label{fig:Desc4commonopen}
\end{figure}

\begin{figure}[H]

{\centering \includegraphics{WriteUp_files/figure-latex/openplot1-1} 

}

\caption{Outcome by Popular Openings\label{Figure5}}\label{fig:openplot1}
\end{figure}

Figures 3.4 and 3.5 show the most played openings and the win
proportions of those openings respectively. The opening names were given
in the dataset with their variations (e.g.~Queen's Gambit: Declined)
where I only want the base opening name. I therefore separated the names
and then dropped the variation. The Sicilian defense being the most
popular is interesting as it is initiated by black as a response to
white's first move of e4 who then may need to change their strategy in
the opening. The opening is usually initiated by white and black has to
adjust their strategy. The power of the Sicilian defense is show in
figure 3.5 with the largest proportion of black winning out of all the
common openings at 50\%. The reason for this is that it partially
eliminates white's first mover advantage as they have to respond in a
way that may not have been their plan. These figures not only highlight
the importance of the opening as one can gain a significant advantage
that can carry through the game but also show that some openings favour
a colour.

\begin{figure}[H]

{\centering \includegraphics{WriteUp_files/figure-latex/ratingoutcomescatter-1} 

}

\caption{Outcome and Ratings\label{Figure6}}\label{fig:ratingoutcomescatter}
\end{figure}

Figure 3.5 shows the influence of the respective players ratings on the
outcome. As expected players with the higher rating will win but around
the 45 degree line there are some exceptions. One would assume that
there would be more white wins at the same or similar rating but at
different levels this seems to change. Between 1000 and 1200 white seems
to win more than black but between 1300 and 1700 it appears that black
wins more than white and above 1700 it appears to be even with the
majority of draws occurring above 2000. This highlights that the
determinants of the outcomes of games may change at different ratings.
Figure 3.6 shows the distribution of the ratings. Both colours have
nearly identical distrbutions which makes sense as a player will have to
play with both colours. The median of black is 1562 and the median of
white is 1567. This distinction is used later to assess whether it is
easier to predict lower or higher rated games. This section has provided
some key insights of the variables of interest and show that they can
have an influence on the outcome.

\begin{figure}[H]

{\centering \includegraphics{WriteUp_files/figure-latex/hist-1} 

}

\caption{Distribution of Ratings\label{Figure7}}\label{fig:hist}
\end{figure}

\hypertarget{methodology}{%
\section{Methodology}\label{methodology}}

\hypertarget{results}{%
\section{Results}\label{results}}

\hypertarget{full-sample}{%
\subsection{Full Sample}\label{full-sample}}

\begin{figure}[H]

{\centering \includegraphics{WriteUp_files/figure-latex/importfull1-1} 

}

\caption{Importance of Features for Untuned Model: Full Sample\label{Figure8}}\label{fig:importfull1}
\end{figure}

\begin{table}[H]
\centering
\begin{tabular}{rrrrrrrrrrr}
  \hline
 & eta & depth & weight & subsample & colsample& gamma & lambda & alpha & rmse & trees \\ 
  \hline
1 & 0.01 & 3.00 & 3.00 & 0.50 & 0.50 & 0.00 & 1.00 & 0.00 & 0.80 & 874.00 \\ 
   \hline
\end{tabular}
\caption{Hypergrid Full Sample} 
\end{table}

\begin{figure}[H]

{\centering \includegraphics{WriteUp_files/figure-latex/importfullsamp2-1} 

}

\caption{Importance of Features for Tuned Model: Full Sample\label{Figure9}}\label{fig:importfullsamp2}
\end{figure}

The initial accuracy after tuning is 62\% on the test sample and 100\%
for the training sample

\begin{table}[ht]
\centering
\begin{tabular}{rrr}
  \hline
 & Sensitivity & Specificity \\ 
  \hline
Class: black & 0.61 & 0.66 \\ 
  Class: draw & 0.12 & 1.00 \\ 
  Class: white & 0.67 & 0.61 \\ 
   \hline
\end{tabular}
\caption{Accuracy Full Sample: Tuned Model} 
\end{table}

After tuning the model has an accuracy of 62\%

\hypertarget{sub-samble-by-elo}{%
\subsection{Sub-samble by ELO}\label{sub-samble-by-elo}}

\hypertarget{bottom-half}{%
\subsubsection{Bottom Half}\label{bottom-half}}

\begin{figure}[H]

{\centering \includegraphics{WriteUp_files/figure-latex/importancelow1-1} 

}

\caption{Feature Importance Untuned Model: Bottom Half\label{Figure11}}\label{fig:importancelow1}
\end{figure}

The initial accuracy before tuning is 59\% on the test sample and 100\%
for the training sample

\begin{table}[H]
\centering
\begin{tabular}{rrrrrrrrrrr}
  \hline
 & eta & depth & weight & subsample & colsample& gamma & lambda & alpha & rmse & trees \\ 
  \hline
1 & 0.01 & 3.00 & 3.00 & 0.50 & 0.50 & 1.00 & 1.00 & 0.00 & 0.80 & 603.00 \\ 

   \hline
\end{tabular}
\caption{Hypergrid Bottom Half} 
\end{table}

\begin{figure}[H]

{\centering \includegraphics{WriteUp_files/figure-latex/importancelow2-1} 

}

\caption{Feature Importance Tuned Model: Bottom Half\label{Figure12}}\label{fig:importancelow2}
\end{figure}

\begin{table}[ht]
\centering
\begin{tabular}{rrr}
  \hline
 & Sensitivity & Specificity \\ 
  \hline
Class: black & 0.55 & 0.65 \\ 
  Class: draw & 0.10 & 1.00 \\ 
  Class: white & 0.65 & 0.56 \\ 
   \hline
\end{tabular}
\caption{Accuracy Bottom Sample: Tuned Model} 
\end{table}

The initial accuracy after tuning is 59\% on the test sample and 100\%
for the training sample

\hypertarget{top-half}{%
\subsubsection{Top Half}\label{top-half}}

\begin{figure}[H]

{\centering \includegraphics{WriteUp_files/figure-latex/impoerancehigh1-1} 

}

\caption{Feature Importance Untuned Model: Top Half\label{Figure13}}\label{fig:impoerancehigh1}
\end{figure}

The initial accuracy before tuning is 58\% on the test sample and 100\%
for the training sample

\begin{table}[H]
\centering
\begin{tabular}{rrrrrrrrrrr}
  \hline
 & eta & depth & weight & subsample & colsample & gamma & lambda & alpha & rmse & trees \\ 
  \hline
1 & 0.01 & 3.00 & 3.00 & 0.50 & 0.50 & 1.00 & 0.00 & 0.00 & 0.81 & 676.00 \\ 

   \hline
\end{tabular}
\caption{Hypergrid Top Half} 
\end{table}

\begin{figure}[H]

{\centering \includegraphics{WriteUp_files/figure-latex/importancehigh2-1} 

}

\caption{Feature Importance Tuned Model: Top Half\label{Figure14}}\label{fig:importancehigh2}
\end{figure}

\begin{table}[ht]
\centering
\begin{tabular}{rrr}
  \hline
 & Sensitivity & Specificity \\ 
  \hline
Class: black & 0.58 & 0.64 \\ 
  Class: draw & 0.11 & 0.99 \\ 
  Class: white & 0.64 & 0.57 \\ 
   \hline
\end{tabular}
\caption{Accuracy Top Sample: Tuned Model} 
\end{table}

The initial accuracy after tuning is 58\% on the test sample and 100\%
for the training sample

\hypertarget{conclusion}{%
\section{Conclusion}\label{conclusion}}

\newpage

\hypertarget{references}{%
\section{References}\label{references}}

\bibliography{Tex/ref}





\end{document}
