\documentclass[12pt,preprint, authoryear]{elsarticle}

\usepackage{lmodern}
%%%% My spacing
\usepackage{setspace}
\setstretch{1.2}
\DeclareMathSizes{12}{14}{10}{10}

% Wrap around which gives all figures included the [H] command, or places it "here". This can be tedious to code in Rmarkdown.
\usepackage{float}
\let\origfigure\figure
\let\endorigfigure\endfigure
\renewenvironment{figure}[1][2] {
    \expandafter\origfigure\expandafter[H]
} {
    \endorigfigure
}

\let\origtable\table
\let\endorigtable\endtable
\renewenvironment{table}[1][2] {
    \expandafter\origtable\expandafter[H]
} {
    \endorigtable
}


\usepackage{ifxetex,ifluatex}
\usepackage{fixltx2e} % provides \textsubscript
\ifnum 0\ifxetex 1\fi\ifluatex 1\fi=0 % if pdftex
  \usepackage[T1]{fontenc}
  \usepackage[utf8]{inputenc}
\else % if luatex or xelatex
  \ifxetex
    \usepackage{mathspec}
    \usepackage{xltxtra,xunicode}
  \else
    \usepackage{fontspec}
  \fi
  \defaultfontfeatures{Mapping=tex-text,Scale=MatchLowercase}
  \newcommand{\euro}{€}
\fi

\usepackage{amssymb, amsmath, amsthm, amsfonts}

\def\bibsection{\section*{References}} %%% Make "References" appear before bibliography


\usepackage[round]{natbib}

\usepackage{longtable}
\usepackage[margin=2.3cm,bottom=2cm,top=2.5cm, includefoot]{geometry}
\usepackage{fancyhdr}
\usepackage[bottom, hang, flushmargin]{footmisc}
\usepackage{graphicx}
\numberwithin{equation}{section}
\numberwithin{figure}{section}
\numberwithin{table}{section}
\setlength{\parindent}{0cm}
\setlength{\parskip}{1.3ex plus 0.5ex minus 0.3ex}
\usepackage{textcomp}
\renewcommand{\headrulewidth}{0.2pt}
\renewcommand{\footrulewidth}{0.3pt}

\usepackage{array}
\newcolumntype{x}[1]{>{\centering\arraybackslash\hspace{0pt}}p{#1}}

%%%%  Remove the "preprint submitted to" part. Don't worry about this either, it just looks better without it:
\makeatletter
\def\ps@pprintTitle{%
  \let\@oddhead\@empty
  \let\@evenhead\@empty
  \let\@oddfoot\@empty
  \let\@evenfoot\@oddfoot
}
\makeatother

 \def\tightlist{} % This allows for subbullets!

\usepackage{hyperref}
\hypersetup{breaklinks=true,
            bookmarks=true,
            colorlinks=true,
            citecolor=blue,
            urlcolor=blue,
            linkcolor=blue,
            pdfborder={0 0 0}}


% The following packages allow huxtable to work:
\usepackage{siunitx}
\usepackage{multirow}
\usepackage{hhline}
\usepackage{calc}
\usepackage{tabularx}
\usepackage{booktabs}
\usepackage{caption}


\newenvironment{columns}[1][]{}{}

\newenvironment{column}[1]{\begin{minipage}{#1}\ignorespaces}{%
\end{minipage}
\ifhmode\unskip\fi
\aftergroup\useignorespacesandallpars}

\def\useignorespacesandallpars#1\ignorespaces\fi{%
#1\fi\ignorespacesandallpars}

\makeatletter
\def\ignorespacesandallpars{%
  \@ifnextchar\par
    {\expandafter\ignorespacesandallpars\@gobble}%
    {}%
}
\makeatother

\newenvironment{CSLReferences}[2]{%
}

\urlstyle{same}  % don't use monospace font for urls
\setlength{\parindent}{0pt}
\setlength{\parskip}{6pt plus 2pt minus 1pt}
\setlength{\emergencystretch}{3em}  % prevent overfull lines
\setcounter{secnumdepth}{5}

%%% Use protect on footnotes to avoid problems with footnotes in titles
\let\rmarkdownfootnote\footnote%
\def\footnote{\protect\rmarkdownfootnote}
\IfFileExists{upquote.sty}{\usepackage{upquote}}{}

%%% Include extra packages specified by user

%%% Hard setting column skips for reports - this ensures greater consistency and control over the length settings in the document.
%% page layout
%% paragraphs
\setlength{\baselineskip}{12pt plus 0pt minus 0pt}
\setlength{\parskip}{12pt plus 0pt minus 0pt}
\setlength{\parindent}{0pt plus 0pt minus 0pt}
%% floats
\setlength{\floatsep}{12pt plus 0 pt minus 0pt}
\setlength{\textfloatsep}{20pt plus 0pt minus 0pt}
\setlength{\intextsep}{14pt plus 0pt minus 0pt}
\setlength{\dbltextfloatsep}{20pt plus 0pt minus 0pt}
\setlength{\dblfloatsep}{14pt plus 0pt minus 0pt}
%% maths
\setlength{\abovedisplayskip}{12pt plus 0pt minus 0pt}
\setlength{\belowdisplayskip}{12pt plus 0pt minus 0pt}
%% lists
\setlength{\topsep}{10pt plus 0pt minus 0pt}
\setlength{\partopsep}{3pt plus 0pt minus 0pt}
\setlength{\itemsep}{5pt plus 0pt minus 0pt}
\setlength{\labelsep}{8mm plus 0mm minus 0mm}
\setlength{\parsep}{\the\parskip}
\setlength{\listparindent}{\the\parindent}
%% verbatim
\setlength{\fboxsep}{5pt plus 0pt minus 0pt}



\begin{document}



\begin{frontmatter}  %

\title{Data Science Methods for Economics and Finance 871 Final Project:
Predicting Chess Outcomes}

% Set to FALSE if wanting to remove title (for submission)




\author[Add1]{Wesley Williams\footnote{Special Thanks to Ruan Geldenhuys
  for his help.}}
\ead{21691126@sun.ac.za}





\address[Add1]{Stellenbosch University, South Africa}


\begin{abstract}
\small{
Chess is one of the most complicated games in the world. Humans on the
other hand are not so complicated and we are the ones playing the game.
This made me ask the question: ``Is it possible to predict a game of
chess based on just on data on the players before the game has started
and just the openings used?'' I employ an extreme gradient boosting
model to answer this question and find that I can only predict the
outcome with just over 60\% accuracy, which is double the chances of
just guessing. You can find my code on my Github page:
\url{https://github.com/wjwilliams/MLPROJ/tree/main/WriteUp}
}
\end{abstract}

\vspace{1cm}





\vspace{0.5cm}

\end{frontmatter}

\setcounter{footnote}{0}


\renewcommand{\contentsname}{Table of Contents}
{\tableofcontents}

%________________________
% Header and Footers
%%%%%%%%%%%%%%%%%%%%%%%%%%%%%%%%%
\pagestyle{fancy}
\chead{}
\rhead{}
\lfoot{}
\rfoot{\footnotesize Page \thepage}
\lhead{}
%\rfoot{\footnotesize Page \thepage } % "e.g. Page 2"
\cfoot{}

%\setlength\headheight{30pt}
%%%%%%%%%%%%%%%%%%%%%%%%%%%%%%%%%
%________________________

\headsep 35pt % So that header does not go over title




\newpage

\hypertarget{introduction}{%
\section{\texorpdfstring{Introduction
\label{Introduction}}{Introduction }}\label{introduction}}

Chess, often considered one of the most popular games globally, has
experienced a surge in popularity in recent years, partly fueled by its
portrayal in popular media such as the acclaimed TV show ``Queen's
Gambit'' and the movie ``Pawn.'' These portrayals, although
fictionalized to varying degrees, shed light on the remarkable life and
achievements of renowned chess player Bobby Fischer. The story of Bobby
Fischer is both inspirational and incredibly sad. During the Cold War,
the United States (US) and the Soviet Union (USSR) engaged in a fierce
competition to assert their global dominance. This rivalry extended
beyond conventional arenas like the space race and nuclear arms race,
encompassing intellectual pursuits such as chess. Chess was revered as
the game of the intellectual elite, and both nations sought to establish
themselves as the preeminent force in this domain, symbolizing their
intellectual superiority. The USSR dominated until Fischer's victory
against world champion Boris Spassky, symbolizing a small triumph for
the US in the Cold War context. This illustrates that chess extends
beyond a mere board game, carrying significant cultural and geopolitical
implications. While the stakes are not as high in the modern era, chess
is still seen as the epitome of intellect.

Chess can be broken down into three stages of the game: the opening, the
middlegame and the endgame. The opening represents the different
strategies of getting all of your pieces into the most optimal positions
on the board to both attack your opponents pieces and defend your own.
This is the key part of a chess game that will be investigated in the
paper. I want to determine if the outcome of a chess game can be
predicted using only the information available about the game and
players before the game begins and the openings employed by the players.
The use of machine learning is crucial in this analysis due to the
complexity of the game. I use an extreme gradient boosting model
(XGBOOST) with three potential outcomes (white winning, black winning or
a draw). The outcome is not binary therefore I cannot use a normal
ordinary least squares as comparison and so I use the probability of
guessing as the baseline. I also subset the data by ELO rating to assess
whether it is easier to predict weaker or stronger players. I find that
the opening is slightly more important for weaker players but the model
is not as accurate as when using the entire sample. The rest of the
paper is structured as follows, in section 2 I describe the data section
3 explores and visualizes the data, section 4 explains the methodology,
section 5 presents the results and section 6 concludes.

\hypertarget{data}{%
\section{Data}\label{data}}

The data was obtained from Kaggle and it includes over 20,000 games
played on the online chess platform ``lichess''. The dataset includes 16
variables but there are only seven variables of interest: 1) the winner
given as white, black or draw; 2) time and increment code which details
the time control as the base time and the additional time per move; 3)
white's ELO rating; 4) black's ELO rating; 5) moves which are all the
moves in the game given in chess notation; 6) opening eco which is a
code that represents the opening that was played and 7) opening ply
which represents the number of moves played in the opening that
corresponds to chess theory. An additional variable of rating difference
was engineered from the perspective of white, which is just white's
rating minus black's rating.

\hypertarget{factor-engineering}{%
\subsection{Factor engineering}\label{factor-engineering}}

Some of the variables need to be wrangled to be used in the model. All
of the numerical variables are sufficient for the model but the
character variables need to be engineered to be included. Firstly, I am
only interested in the first five moves of the game, I therefore expand
the moves variable so that I have a variable for each of the first five
moves played by each player and disregard the rest. The moves are then
converted from chess notation to a unique numerical factor for each
piece moving to each co-ordinate on the board. Secondly, I separate the
increment variable into the base time and the increment for each move
and ensure both are numerical factors. Lastly, I assign a unique
numerical value to each of the unique openeing codes to ensure
comparability between the training as testings
samples\footnote{I did attempt one-hot encoding but I had issues with the differences in lengths between the training and testing samples}.
All the variables are therefore sufficiently engineered to be used in
the model.

\hypertarget{exploratory-data-analysis}{%
\section{Exploratory Data Analysis}\label{exploratory-data-analysis}}

In this section I attempt to explore and visualize the data to gain
insights into the patterns that emerge with respect all the features and
the the target.

\begin{figure}[H]

{\centering \includegraphics{WriteUp_files/figure-latex/commove1-1} 

}

\caption{Most Common Opening Moves by Colour\label{Figure1}}\label{fig:commove1}
\end{figure}

\begin{figure}[H]

{\centering \includegraphics{WriteUp_files/figure-latex/firstmovepropwins-1} 

}

\caption{Proportions of Outcomes by White First Move\label{Figure2}}\label{fig:firstmovepropwins}
\end{figure}

Figure 3.1 shows the most common first move for each colour. It is no
surprise that central pawn moves are the most common as controlling the
center of the board is instrumental in the opening phase of a chess
game. Figure 3.2 shows the outcome proportions for white's first move
and it shows that if white claims the center they gain an advantage and
black needs to respond. The move e3 instead allows black to claim the
center and white loses its first mover advantage. This highlights that
mistakes early on in the openings have consequences that last throughout
the entire game.

\begin{figure}[H]

{\centering \includegraphics{WriteUp_files/figure-latex/Descript3wins-1} 

}

\caption{Outcomes by Time Controls\label{Figure3}}\label{fig:Descript3wins}
\end{figure}

Figure 3.3 shows the differences in outcome according to different time
controls. It presents the top 5 time controls that are played with and
without time increments. There are no significant differences in wins
between whether a game has time increments or not. The only real
difference is that a draw is more likely with larger time controls and
no time increments.

\begin{figure}[H]

{\centering \includegraphics{WriteUp_files/figure-latex/Desc4commonopen-1} 

}

\caption{Most Popular Openings\label{Figure4}}\label{fig:Desc4commonopen}
\end{figure}

\begin{figure}[H]

{\centering \includegraphics{WriteUp_files/figure-latex/openplot1-1} 

}

\caption{Outcome by Popular Openings\label{Figure5}}\label{fig:openplot1}
\end{figure}

Figures 3.4 and 3.5 show the most played openings and the win
proportions of those openings respectively. The opening names were given
in the dataset with their variations (e.g.~Queen's Gambit: Declined)
where I only want the base opening name. I therefore separated the names
and then dropped the variation. The Sicilian defense being the most
popular is interesting as it is initiated by black as a response to
white's first move of e4 who then may need to change their strategy in
the opening. The opening is usually initiated by white and black has to
adjust their strategy. The power of the Sicilian defense is show in
figure 3.5 with the largest proportion of black winning out of all the
common openings at 50\%. The reason for this is that it partially
eliminates white's first mover advantage as they have to respond in a
way that may not have been their plan. These figures not only highlight
the importance of the opening as one can gain a significant advantage
that can carry through the game but also show that some openings favour
a colour.

\begin{figure}[H]

{\centering \includegraphics{WriteUp_files/figure-latex/ratingoutcomescatter-1} 

}

\caption{Outcome and Ratings\label{Figure6}}\label{fig:ratingoutcomescatter}
\end{figure}

Figure 3.5 shows the influence of the respective players ratings on the
outcome. As expected players with the higher rating will win but around
the 45 degree line there are some exceptions. One would assume that
there would be more white wins at the same or similar rating but at
different levels this seems to change. Between 1000 and 1200 white seems
to win more than black but between 1300 and 1700 it appears that black
wins more than white and above 1700 it appears to be even with the
majority of draws occurring above 2000. This highlights that the
determinants of the outcomes of games may change at different ratings.
Figure 3.6 shows the distribution of the ratings. Both colours have
nearly identical distrbutions which makes sense as a player will have to
play with both colours. The median of black is 1562 and the median of
white is 1567. This distinction is used later to assess whether it is
easier to predict lower or higher rated games. This section has provided
some key insights of the variables of interest and show that they can
have an influence on the outcome.

\begin{figure}[H]

{\centering \includegraphics{WriteUp_files/figure-latex/hist-1} 

}

\caption{Distribution of Ratings\label{Figure7}}\label{fig:hist}
\end{figure}

\hypertarget{methodology}{%
\section{Methodology}\label{methodology}}

I use a gradient boosting model with the winner of the game as the
target and the features being both players ratings, their rating
difference, the opening used, the number of moves played in the opening,
the time controls and the first five moves of each player. I stratified
the sampling by the opening to ensure that as many openings were
included in both the training and test split. This also means that the
moves were stratified as the moves correspond to the opening. Gradient
boosting employs a sequence of shallow trees that learn from the
previous tree and makes improvements. I used the XGBOOST package and due
to the target variable not being binary and because I had more than 53
categories for the opening feature which the random forrest package in R
could not handle. I used the ``multi:softmax'' objective function to be
able to predict three types of outcomes and the evaluation metric was a
multiclass log loss which evaluates the logarithm of the predicted
probabilities for each class and aggregates them across all instances
and is calculated as follows.

\begin{equation}
\text{{mlogloss}} = -\frac{1}{n} \sum \left[ y \cdot \log(p) + (1-y) \cdot \log(1-p) \right]
\end{equation}

The methods used follow Boehmke \& Greenwell
(\protect\hyperlink{ref-homl2019}{2019}). I first trained and tested the
model with default parameters before employing a grid search that uses
cross validation to determine the optimal value of the learning rate
(\(\eta\)), the minimum loss reduction(\(\gamma\)), the L2
regularization term (\(\lambda\)) and the number of trees. I then used
the results of the grid search to hyper parameter tune the model. I then
used the importance of each factor and a confusion matrix to interpret
and assess the model. I then subset the data using the median ratings
and used same method and steps to assess if it is easier or more
difficult to predict stronger or weaker players.

\hypertarget{results}{%
\section{Results}\label{results}}

This section presents the results of all six models. I present the
importance of the features of the models with the default parameters to
see how the importance changes after hyperparameter tuning. For the
tuned models I also present the sensitivity and specificity calculated
through a confusion matrix.

\hypertarget{full-sample}{%
\subsection{Full Sample}\label{full-sample}}

\begin{figure}[H]

{\centering \includegraphics{WriteUp_files/figure-latex/importfull1-1} 

}

\caption{Importance of Features for Untuned Model: Full Sample\label{Figure8}}\label{fig:importfull1}
\end{figure}

Figure 5.1 shows that, as expected, ratings features are the most
important features, combined they account for 41\% and are the three
most important features of the model. The opening that is played is only
the fifth most important factor at only 6\%. The results also show that
the moves increase in importance with both white and black's first moves
being the least important features and every subsequent move becomes
more important. Black's moves are also shown to be more important to the
model than white's moves with black's fifth move being the fourth most
important feature at 8\% which is interesting as it means it is more
important than the opening. The model has an accuracy of 100\% on the
training set and just 60\% on the testing set so it may be over fitted.
I then used hyper parameter tuning to attempt to address this issue.
Table 5.1 below presents the most optimal parameters provided by the
grid search. The results suggest the learning rate (\(\eta\)) should be
reduced from 0.1 to 0.01. The parameters \(\alpha\), \(\gamma\) and
\(\lambda\) do not change and are still zero. The tree depth should be
reduced from 6 to 3, which should also help solve the issue of over
fitting. These parameters are used to obtain the final model and the
results are presented below.

\begin{table}[H]
\centering
\begin{tabular}{rrrrrrrrrrr}
  \hline
 & $\eta$ & Depth & Weight & Subsample & Colsample& $\gamma$ & $\lambda$ & $\alpha$ & RMSE & Trees \\ 
  \hline
1 & 0.01 & 3.00 & 3.00 & 0.50 & 0.50 & 0.00 & 0.00 & 0.00 & 0.80 & 874.00 \\ 
   \hline
\end{tabular}
\caption{Hypergrid Full Sample} 
\end{table}

\begin{figure}[H]

{\centering \includegraphics{WriteUp_files/figure-latex/importfullsamp2-1} 

}

\caption{Importance of Features for Tuned Model: Full Sample\label{Figure9}}\label{fig:importfullsamp2}
\end{figure}

\begin{table}[ht]
\centering
\begin{tabular}{rrr}
  \hline
 & Sensitivity & Specificity \\ 
  \hline
Class: black & 0.61 & 0.66 \\ 
  Class: draw & 0.12 & 1.00 \\ 
  Class: white & 0.67 & 0.61 \\ 
   \hline
\end{tabular}
\caption{Accuracy Full Sample: Tuned Model} 
\end{table}

Figure 5.2 and table 5.2 present the results of the tuned model. The
tuning of parameters has improved the accuracy on the testing set to
61.66\% but again the accuracy on the training set is still 99.98\%
which means that even after tuning the model still suffers from over
fitting. The importance of the rating features increased to 68.92\%. The
lower learning rate results in better accuracy but it seems that the
increased accuracy is driven by the ratings features rather than moves.
The opening now becomes the most important feature after ratings which
shows that it can influence the game but its importance decreases to
just 3.73\%. The time controls also become relatively more important
after tuning but similarly to the opening it percentage of importance
also decreases.

Table 5.2 is an extract of the model's confusion matrix where
sensitivity is the percentage of white wins accurately predicted and
specificity is the percentage of white not winning accurately predicted.
The table shows that the model predicts white the best with a
sensitivity and specificity of 66.67\% and 60.92\% respectively. It
predicts black almost as well with a sensitivity and specificity of 61\%
and 66.13\% respectively, unlike for white the specificity is larger
than the sensitivity meaning it predicts black not winning more
accurately than black winning. The model struggles the most to predict a
draw with a sensitivity of just 11.74\%. The accuracy is still nearly
double that of just guessing which would be 33\%.

\hypertarget{sub-samble-by-elo}{%
\subsection{Sub-samble by ELO}\label{sub-samble-by-elo}}

The sample was then subset using the medians of ratings calculated in
the exploratory data analysis section. I use the same methodology as
above for both subsets of the sample.

\hypertarget{bottom-half}{%
\subsubsection{Bottom Half}\label{bottom-half}}

\begin{figure}[H]

{\centering \includegraphics{WriteUp_files/figure-latex/importancelow1-1} 

}

\caption{Feature Importance Untuned Model: Bottom Half\label{Figure11}}\label{fig:importancelow1}
\end{figure}

Figure 5.3 shows the feature importance of the untuned model which has
an accuracy of 58\% which is 2\% lower than the full sample. Again
ratings feasures are the most important and combined contribute 34.33\%
to the model. The relative importance of openings is much lower than in
the full sample as it is only the eighth most important factor now with
both white and black's fourth and fifth moves contributing more to the
model. The results of the grid search are presented below in table 5.3
and the results suggest the same parameters as for the full sample with
the exception of \(\lambda\) and \(\gamma\) which are now both 1. The
results of the hyper parameter tuned model are presented below.

\begin{table}[H]
\centering
\begin{tabular}{rrrrrrrrrrr}
  \hline
 & $\eta$ & Depth & Weight & Subsample & Colsample& $\gamma$ & $\lambda$ & $\alpha$ & RMSE & Trees \\  
  \hline
1 & 0.01 & 3.00 & 3.00 & 0.50 & 0.50 & 1.00 & 1.00 & 0.00 & 0.80 & 603.00 \\ 

   \hline
\end{tabular}
\caption{Hypergrid Bottom Half} 
\end{table}

\begin{figure}[H]

{\centering \includegraphics{WriteUp_files/figure-latex/importancelow2-1} 

}

\caption{Feature Importance Tuned Model: Bottom Half\label{Figure12}}\label{fig:importancelow2}
\end{figure}

\begin{table}[ht]
\centering
\begin{tabular}{rrr}
  \hline
 & Sensitivity & Specificity \\ 
  \hline
Class: black & 0.55 & 0.65 \\ 
  Class: draw & 0.10 & 1.00 \\ 
  Class: white & 0.65 & 0.56 \\ 
   \hline
\end{tabular}
\caption{Accuracy Bottom Sample: Tuned Model} 
\end{table}

The tuned model still performs worse than the full sample even after
tuning with an accuracy of only 58.61\%. This makes sense as weaker
players will not stick to rigid openings as they do not know much chess
theory. This makes them more unpredictable and therefore it is expected
that the model performs worse. The model still seems to be over fitted
as the accuracy on the training set is still 99.98\%.

Figure 5.4 presents the importance of features in the tuned model. The
decrease in the learning rate parameter does not increase the importance
of the ratings features as much as the tuned model for the full sample
as it only increases to 39.72\%. The type of opening used now increases
in both relative and absolute importance. It is now the fourth most
important feature and its importance increased to 6.36\% from 6\%. The
decrease in the learning rate helped to identify that even for weaker
players the opening used is the most important feature after rating
features.

Table 5.4 breaks down the models accuracy and again it best predicts
white, then black and still struggles with predicting a draw. All values
of sensitivity and specificity have decreased compared to the full
sample so it can be concluded isolating the weaker players does not
improve the models accuracy.

\hypertarget{top-half}{%
\subsubsection{Top Half}\label{top-half}}

\begin{figure}[H]

{\centering \includegraphics{WriteUp_files/figure-latex/impoerancehigh1-1} 

}

\caption{Feature Importance Untuned Model: Top Half\label{Figure13}}\label{fig:impoerancehigh1}
\end{figure}

Figure 5.3 shows the feature importance of the untuned model which has
an accuracy of 57.68\% which is 2\% lower than the full sample. This is
interesting as I would have expected the stronger players to be easier
to predict as they play more rigid chess and follow chess principles
more stringently. Again ratings features are the most important and
combined contribute 38.19\% to the model. The importance of the opening
for the stronger players is apparent already as it is the fourth most
important factor at 7.49\%. The ordering of the importance of the rest
of the does not change compared to both other samples. Table 5.5 shows
the results of the grid search and the parameter values are the same as
for the bottom sample with the exception of \(\lambda\) which is 0,
similar to the full sample. The parameters from the grid search are then
used to hyper parameter tune the model and the results are presented
below.

\begin{table}[H]
\centering
\begin{tabular}{rrrrrrrrrrr}
  \hline
 & $\eta$ & Depth & Weight & Subsample & Colsample& $\gamma$ & $\lambda$ & $\alpha$ & RMSE & Trees \\ 
  \hline
1 & 0.01 & 3.00 & 3.00 & 0.50 & 0.50 & 1.00 & 0.00 & 0.00 & 0.81 & 676.00 \\ 

   \hline
\end{tabular}
\caption{Hypergrid Top Half} 
\end{table}

\begin{figure}[H]

{\centering \includegraphics{WriteUp_files/figure-latex/importancehigh2-1} 

}

\caption{Feature Importance Tuned Model: Top Half\label{Figure14}}\label{fig:importancehigh2}
\end{figure}

\begin{table}[ht]
\centering
\begin{tabular}{rrr}
  \hline
 & Sensitivity & Specificity \\ 
  \hline
Class: black & 0.58 & 0.64 \\ 
  Class: draw & 0.11 & 0.99 \\ 
  Class: white & 0.64 & 0.57 \\ 
   \hline
\end{tabular}
\caption{Accuracy Top Sample: Tuned Model} 
\end{table}

The hyper parameter tuning does not increase the overall accuracy by
much and it is still 57.68\% on the test set and 99.98\% for the
training set. This is puzzling as not only is there no improvement, it
also worse than for the bottom half sample. Over fitting is still an
issue, the hyper parameter tuning does not seem to solve. Figure 5.6
shows the importance of the features and the importance of the ratings
features increased to 55.67\% which is much larger than for the bottom
sample. Simalarly the importance of the opening for stronger players is
only 4.66\%.

Table 5.6 presents the results from the confusion matrix. It shows that
the model predicts black winning and losing more accurately than for the
weaker players but predicts white winning and losing less accurately
than for the weaker players. Its accuracy and predicting the game to be
a draw also decreases. While my assumption going in was that it would be
easier to predict the outcome for stronger players was incorrect it
highlights how complex the game of chess is and that stronger players
seem to rely less on the openings than weaker players.

\hypertarget{conclusion}{%
\section{Conclusion}\label{conclusion}}

Chess is a complicated game and trying to predict the outcome based just
on the opening does not work. My results suffer from over fitting which
is a large limitation of the investigation. The results show that the
type of opening used is a determinant of the outcome but is minuscule
compared to the ratings features. The importance of the opening also
seems to decrease for players that are stronger, indicating that they
rely less on specific opening strategies and may adapt their gameplay
based on the opponent's moves and overall game dynamics. Subsetting the
data based on player ratings, either by considering the bottom or top
half, resulted in models that were less accurate compared to the
analysis on the full sample. This suggests that having a diverse range
of player ratings in the dataset improves the model's accuracy, likely
due to the increased variability and complexity introduced by players of
different skill levels.

In conclusion, while the opening moves do play a role in determining the
outcome of a chess game, they are overshadowed by the ratings features
of the players. To build a more accurate prediction model, it is
important to consider a wide range of player ratings and incorporate
other relevant features beyond just the opening moves. Chess remains a
challenging game to predict, highlighting the intricate nature of the
game and the need for comprehensive models that encompass various
aspects of player skill and strategy.

\newpage

\hypertarget{references}{%
\section*{References}\label{references}}
\addcontentsline{toc}{section}{References}

\hypertarget{refs}{}
\begin{CSLReferences}{1}{0}
\leavevmode\vadjust pre{\hypertarget{ref-homl2019}{}}%
Boehmke, B. \& Greenwell, B.M. 2019. \emph{Hands-on machine learning
with r}. CRC press.

\end{CSLReferences}

\bibliography{Tex/ref}





\end{document}
